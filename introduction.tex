Below is the composite mean atmospheric circulation for each of the Antarctic Cloud Mass Meridional Transport (CMMT) regions over the entire study period (1 November 1992 -- 31 October 2012).

For this analysis, 6-hourly, 500hPa zonal and meridional wind data from the European Centre for Medium-Range Weather Forecasts Interim Reanalysis \citep[ERA-Interim;][]{Dee2011} was downloaded for the period 1979--2016. Daily mean fields were calculated from the 6-hourly wind data and these were used to calculate the 500hPa streamfunction anomaly (i.e. the mean value for each day over the period 1979--2016 was calculated to produce a daily climatology, and then the corresponding climatological mean value was subtracted at each data time to obtain the anomaly).   

When calculating the composite mean field for a given region, all days for which a CMMT event was recorded (i.e. at any time of that day) were used, regardless of whether the event was designated as a skirting event or not. While this is a rather coarse approach, I suspect a more detailed approach (e.g. it would be more correct to use the 6 hourly data and try to break down skirting events according to time spent in each region) would yield very similar results.

First impressions:
