Below is the composite mean atmospheric circulation for each of the Antarctic Cloud Mass Meridional Transport (CMMT) regions over the entire study period (1 November 1992 -- 31 October 2012).

For this analysis, 6-hourly, 500 hPa zonal and meridional wind data from the European Centre for Medium-Range Weather Forecasts Interim Reanalysis \citep[ERA-Interim;][]{Dee2011} was downloaded for the period 1979--2016. Daily mean fields were calculated from the 6-hourly wind data and these were used to calculate the 500 hPa streamfunction anomaly (i.e. the mean value for each day over the period 1979--2016 was calculated to produce a daily streamfunction climatology, and then the corresponding climatological mean value was subtracted at each data time to obtain the streamfunction anomaly).   

When calculating the composite mean field for a given region, all days for which a CMMT event was recorded (i.e. at any time of the day) were used, regardless of whether the event was designated as a skirting event or not. While this is a rather coarse approach, I suspect a more detailed approach (e.g. it would be more correct to use the 6-hourly data and try to break down skirting events according to time spent in each region) would yield similar results.

First impressions on recognizable structures in the composites: 
\begin{itemize}
\item The composite mean circulation for Ellsworth Land (Figure \ref{fig:ellsworth}) resembles the Pacific-South American (PSA) pattern. The PSA pattern has traditionally been linked to convection in the tropical Pacific (and thus is routinely introduced as a mechanism by which ENSO can influence the high southern latitudes), however a couple of recent studies \citep{Irving2016,OKane2017} have challenged this assumption (they find it is instead an intrinsic feature of the mid-to-high latitude circulation that is largely independent of tropical forcing). This might explain why there was no strong association between CMMT events in Ellsworth Land and ENSO indices.
\item There are a limited number of seasons/regions that show a fairly coordinated zonal wave number 4 pattern around the hemisphere (e.g. Queen Maud Land in MAM; Figure \ref{fig:queen_maud}). This wavenumber 4 variability is to be expected (Figure 4b of \citet{Irving2015} shows that zonal wavenumber 4 is dominant at daily timescales), but I'm actually not sure about the relationship between such coordinated daily timescale patterns and the well known monthly timescale features like the ZW3 pattern. (i.e. does a strong monthly mean ZW3 pattern consist of lots of daily mean fields that have a coordinated wavenumber 4 pattern? This might be true - I could possibly look into that if need be.)
\item In many seasons/regions (particularly for the annual plots) the streamfunction anomalies are strong in the vicinity of the region of interest but weak further afield (I've used an exaggerated number of contours in the plots to try and make sure we don't miss any spatial structures, but if you use a more reasonable contour interval it becomes clearer that in many cases the far field anomalies are rather weak). This might be an interesting finding in itself, in that it suggests that there isn't an obvious large-scale dynamical climate phenomena (e.g. ENSO, SAM, ZW3, PSA pattern, etc) associated with events in that season/region (i.e. perhaps local factors are much more important - future work could try and identify some of those local factors?)  
\item It might be worth looking at the individual daily fields associated with some of the events to get a feel for if the composites are representative (i.e. if there is a lot of variability between events the composite mean might not be particularly representative)
\end{itemize}
